\RequirePackage[ngerman=ngerman-x-latest]{hyphsubst}
\documentclass[ngerman]{tudscrreprt}

%\usepackage{selinput}
\usepackage[utf8]{inputenc}
\usepackage[USenglish]{babel}
\usepackage[T1]{fontenc}
\usepackage{scrhack}
\usepackage{tudscrsupervisor}
\usepackage{color}
\usepackage{transparent}
\usepackage{url}
\usepackage{graphicx,subfig}
\graphicspath{{img/}}
\usepackage{wrapfig}
\usepackage{csquotes}
\usepackage[backend=biber,style=alphabetic]{biblatex}

% Literaturverzeichnis
\addbibresource{bibHauptseminar.bib}

\usepackage[linkcolor=black]{hyperref}

\begin{document}
\faculty{Fakultät Elektrotechnik und Informationstechnik}
\institute{Institut für Nachrichtentechnik}
\chair{Deutsche Telekom Professur für Kommunikationsnetze}
\date{\today}
\title{Compressed Compute-and-Forward mit korrelierten Audiosignalen}
\subject{}

\author{%
	Lucas Weber
	\matriculationnumber{xxxxx}
	\dateofbirth{10.01.1995}
	\and%
	Raphael Hildebrand
	\matriculationnumber{xxxxxx}
	\dateofbirth{01.08.1994}
	\and%
	Florian Roth
	\matriculationnumber{xxxxx}
	\dateofbirth{13.01.1994}
	\and%
	Orell Garten
	\matriculationnumber{3948375}
	\dateofbirth{09.07.1993}
}
\matriculationyear{2013}
\supervisor{Carsten Hermann}
\professor{Dr.-Ing. Rico Radeke}

\maketitle

\include{task}


\TUDoption{abstract}{multiple,section}
\begin{abstract}
 Dies ist der deutschsprachige Teil der Zusammenfassung, in dem die
 Motivation sowie der Inhalt der nachfolgenden wissenschaftlichen
 Abhandlung kurz dargestellt werden.
\nextabstract[english]
 This is the english part of the summary, in which the motivation and
 the content of the following academic treatise are briefly presented.
\end{abstract}

\confirmation

\tableofcontents

\listoffigures

\include{Hauptteil}

\printbibliography

\end{document}
