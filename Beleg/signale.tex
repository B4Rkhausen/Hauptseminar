Wir haben für die Aufnahme der Signale viel verschiedene Situationen ausgesucht, um ein möglichst breites Spektrum an Raum-Effekten zu erhalten. Aufnahmeorte waren beispielsweise der Platz vor dem HSZ, die Wiese zwischen Physik- und Mathematikgebäude, sowie der Trefftzbau. Außerdem wurde in einer Wohnung gemessen, um Effekte von schallabsorbierenden Stoffen wie Teppich oder Bett zu erhalten. Soweit mgölich, haben wir die natürliche Geräuschkulisse am jeweiligen Ort eingefangen. Zusätzlich dazu wurde ein definiertes Signal mittels eines Lautsprechers erzeugt, um Direktschall zu nutzen. Bei diesen Aufnahmen sollten die Effekte des Raumes am deutlichsten hervortreten.
\subsection{Beispielsignal}
\subsection{Probleme bei der Signalauswahl}
