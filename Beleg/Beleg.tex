\documentclass[a4paper,11pt]{article}
\usepackage[T1]{fontenc}
\usepackage[utf8]{inputenc}
\usepackage{lmodern}
\usepackage{listings}
\usepackage{color}
\usepackage{graphicx}

\lstdefinelanguage
   [x64]{Assembler}     % add a "x64" dialect of Assembler
   [x86masm]{Assembler} % based on the "x86masm" dialect
   % with these extra keywords:
   {morekeywords={CDQE,CQO,CMPSQ,CMPXCHG16B,JRCXZ,LODSQ,MOVSXD, %
                  POPFQ,PUSHFQ,SCASQ,STOSQ,IRETQ,RDTSCP,SWAPGS, %
                  rax,rdx,rcx,rbx,rsi,rdi,rsp,rbp, %
                  r8,r8d,r8w,r8b,r9,r9d,r9w,r9b}} % etc.

\lstset{ % General setup for the package
	language=[x64]Assembler,
	breaklines=true,
	basicstyle=\small\sffamily,
	numbers=left,
 	numberstyle=\tiny,
	frame=tb,
	tabsize=4,
	%columns=fixed,
	showstringspaces=false,
	showtabs=false,
	keepspaces,
	commentstyle=\color{black},
	keywordstyle=\color{blue}
}

\title{Hauptseminar Kommunikationssysteme \newline Compressed Compute-and-Forward mit korrelierten Audiosignalen}
\date{13.07.2016}
\author{Florian Roth, Raphael Hildebrand, Lucas Weber, Orell Garten}

\begin{document}

\pagenumbering{gobble}
\maketitle
\newpage
\tableofcontents
\newpage
\pagenumbering{arabic}

\section{Aufgabenstellung}

\section{Motivation}

\section{Theoretische Vorbetrachtung}
\subsection{Korrelationsfunktion}
\subsection{Parameter}
\subsection{Gauß-Regression}
\section{Programm}
\subsection{Aufbau des Programms}
\subsection{Probleme} 
\subsection{Format der Lösung}

\section{Signalauswahl}
\subsection{Beispielsignal}

\begin{appendix}

\end{appendix}

\end{document}
